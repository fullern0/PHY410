\documentclass[aps,12pt,nobalancelastpage,amsmath,amssymb,
nofootinbib]{revtex4}

\usepackage{graphics}      % standard graphics specifications
\usepackage{graphicx}      % alternative graphics specifications
\usepackage{longtable}     % helps with long table options
\usepackage{url}           % for on-line citations
\usepackage{bm}            % special 'bold-math' package
\usepackage{array}   
\usepackage{float}
\usepackage{mathtools}
\DeclarePairedDelimiter\ceil{\lceil}{\rceil}
\DeclarePairedDelimiter\floor{\lfloor}{\rfloor}

\usepackage{fancyhdr}
\fancyhf{} % clear all header and footers
\renewcommand{\headrulewidth}{0pt} % remove the header rule
\cfoot{\thepage}
\pagestyle{fancy}

\begin{document}
The largest number which can be stored within $m$ bits is $2^{m}-1$. If the series $f_{n}=f^{2}_{n-1}$ is to be computed, the number of steps $n$ is limited by the size of the variable storing $f_{n}$. If $f_{0}=2$, then
\begin{equation}
f_{n}=2^{2^{n}}.
\end{equation}
Since $f_{n}\leq 2^{m}-1$, it follows that $2^{n}<m$. The largest $n$ can be given a variable $m$ bits long is then $\floor{\text{log}_{2}(m-1)}$.\\\\
A$)$ $\floor{\text{log}_{2}(8-1)}=2$\\\\
B$)$ $\floor{\text{log}_{2}(16-1)}=3$\\\\
C$)$ $\floor{\text{log}_{2}(32-1)}=4$

\end{document}