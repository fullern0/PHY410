\documentclass[aps,12pt,nobalancelastpage,amsmath,amssymb,
nofootinbib]{revtex4}

\usepackage{graphics}      % standard graphics specifications
\usepackage{graphicx}      % alternative graphics specifications
\usepackage{longtable}     % helps with long table options
\usepackage{url}           % for on-line citations
\usepackage{bm}            % special 'bold-math' package
\usepackage{array}   
\usepackage{float}

\usepackage{fancyhdr}
\fancyhf{} % clear all header and footers
\renewcommand{\headrulewidth}{0pt} % remove the header rule
\cfoot{\thepage}
\pagestyle{fancy}

\begin{document}
The two's compliment is a convenient way to store negative numbers for mathematical operations. The two's compliment for a positive number is itself. To find the two's compliment of a negative number $-x$, subtract one from $x$ then bit invert the result. I am using 16 bits to represent each number.\\\\
A$)$ $10\rightarrow 10$\\\\
B$)$ $436\rightarrow 436$\\\\
C$)$ $1024\rightarrow 1024$\\\\
D$)$ $-13$\\
$13-1=12$\\
$12\rightarrow 0000000000001100$\\
$\sim 0000000000001101=1111111111110011$\\
$1111111111110011\rightarrow 65523$\\\\
E$)$ $-1023$\\
$1023-1=1022$\\
$1022\rightarrow 0000001111111110$\\
$\sim 0000001111111110=1111110000000001$\\
$1111110000000001\rightarrow 64513$\\\\
F$)$ $-1024$\\
$1024-1=1023$\\
$1023\rightarrow 0000001111111111$\\
$\sim 0000001111111111=1111110000000000$\\
$1111110000000000\rightarrow 64512$

\end{document}